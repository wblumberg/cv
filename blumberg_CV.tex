%%%%%%%%%%%%%%%%%%%%%%%%%%%%%%%%%%%%%%%%%
% Long Sectioned Curriculum Vitae
% LaTeX Template
% Version 1.1 (9/12/12)
%
% This template has been downloaded from:
% http://www.latextemplates.com
%
% Original author:
% Rensselaer Polytechnic Institute (http://www.rpi.edu/dept/arc/training/latex/resumes/)
%
% Important note:
% This template requires the res.cls file to be in the same directory as the
% .tex file. The res.cls file provides the resume style used for structuring the
% document.
%
%%%%%%%%%%%%%%%%%%%%%%%%%%%%%%%%%%%%%%%%%

%----------------------------------------------------------------------------------------
%	PACKAGES AND OTHER DOCUMENT CONFIGURATIONS
%----------------------------------------------------------------------------------------

\documentclass[10pt]{res} % Use the res.cls style, the font size can be changed to 11pt or 12pt here

\usepackage{helvet} % Default font is the helvetica postscript font
%\usepackage{newcent} % To change the default font to the new century schoolbook postscript font uncomment this line and comment the one above
\usepackage{color, hyperref}
\definecolor{darkblue}{rgb}{0.0,0.0,0.7}
\hypersetup{colorlinks,breaklinks,linkcolor=darkblue,urlcolor=darkblue,anchorcolor=darkblue,citecolor=darkblue}

\newsectionwidth{0pt} % Stops section indenting

\newcommand{\MyName}[1]{ % Name
		\LARGE \usefont{OT1}{phv}{b}{n} #1
		\normalfont}
		
\newcommand{\MySlogan}[1]{ % Slogan (optional)
		\large \usefont{OT1}{phv}{m}{n} \textit{#1}
		\normalsize \normalfont}

\newcommand{\contact}[1]{ % Slogan (optional)
		\small \usefont{OT1}{phv}{m}{n} #1
		\normalsize \normalfont}

\pagestyle{plain}
\pagenumbering{arabic}

\begin{document}


%----------------------------------------------------------------------------------------
%	YOUR NAME AND ADDRESS(ES) SECTION
%----------------------------------------------------------------------------------------

\MyName{William Gregory Blumberg, Ph.D.}\hfill \small Updated June 2018\vspace{3pt}\\
\MySlogan{Curriculum Vitae} \vspace{2pt}\\
CIMMS/OU School of Meteorology\\
\small
120 David L. Boren Blvd.\\
Norman, OK 73072\\
Email: wblumberg@ou.edu
\normalsize
\vspace{-10pt} % Gap between title and text

%----------------------------------------------------------------------------------------
\noindent\makebox[\linewidth]{\rule{\textwidth}{1pt}} 
\vspace{-25pt} % Gap between title and text

\begin{resume}

%----------------------------------------------------------------------------------------
%	OBJECTIVE SECTION
%----------------------------------------------------------------------------------------

%\section{\centerline{OBJECTIVE}}

%\vspace{8pt} % Gap between title and text

%A position in Personnel Administration utilizing skills in recruiting, training and compensation.\\ 

%----------------------------------------------------------------------------------------
%	EDUCATION SECTION
%----------------------------------------------------------------------------------------

\section{\centerline{EDUCATION}} 

\vspace{8pt} % Gap between title and text

{\sl Doctor of Philosophy}, 
Meteorology \hfill June 2014 - May 2018 \\
University of Oklahoma, Norman, OK \\ 
Dissertation: \href{https://shareok.org/handle/11244/299948}{Observations and Simulations of Moisture Changes Occurring Around Sunset at the ARM Climate Research Facility}\\
Advisors: \href{https://esrl.noaa.gov/gsd/mdb/turner.html}{Dr. Dave Turner} (NOAA ESRL) and \href{http://arctic.som.ou.edu/scavallo/}{Dr. Steven Cavallo} (OU) \\

{\sl Master of Science}, 
Meteorology \hfill August 2011 - August 2013 \\
University of Oklahoma, Norman, OK \\ 
Thesis: Developing a Statistical Thermodynamic Retrieval for Ground-Based Infrared Spectrometers \\
Advisors: Drs. \href{http://weather.ou.edu/~oubliss/klein}{Petra Klein} (OU) and \href{https://esrl.noaa.gov/gsd/mdb/turner.html}{Dave Turner} (NOAA ESRL) \\
 
{\sl Bachelor of Science}, Meteorology  \hfill August 2007 - May 2011 \\
University of Oklahoma, Norman, OK \\
Minor - Mathematics \\
Advisor: \href{http://spider.ags.ou.edu/~kkloesel/}{Dr. Kevin Kloesel} (OU)
%----------------------------------------------------------------------------------------
 
\vspace{0.2in} % Some whitespace between sections

%----------------------------------------------------------------------------------------
%	RESEARCH EXPERIENCE SECTION
%----------------------------------------------------------------------------------------

\section{\centerline{RESEARCH EXPERIENCE}} 

\vspace{8pt} % Gap between title and text

{\sl Graduate Research Assistant} \hfill June 2014 -- May 2018 \\
Cooperative Institute for Mesoscale Meteorology Studies (CIMMS)/University of Oklahoma, Norman, OK
\begin{itemize} \itemsep 2pt % Reduce space between items
\item Assisted in developing ground-based observation strategies for the \href{https://www.eol.ucar.edu/field_projects/pecan}{Plains Elevated Convection At Night} (PECAN) field project.  Also participated in deployments of the \href{http://www.nssl.noaa.gov/users/dturner/public_html/CLAMPS/}{Collaborative Lower Atmosphere Mobile Profiling System} (CLAMPS).
\item Performed research designed to improve the understanding of how radiosondes, Atmospheric Emitted Radiance Interferometer (AERI), and lidar instruments can be used to diagnose the ingredients relevant to deep moist convection.
\item Designed and performed experiments using high-temporal resolution, ground-based instrumentation to improve our understanding of the processes that drive the rapid evolution of instability, moisture, and shear during the Southern Great Plains afternoon to evening transition.
%\item Began developing 
\end{itemize}

{\sl Research Fellow} \hfill August 2013 -- June 2014 \\
CIMMS, Norman, OK
\begin{itemize} \itemsep 2pt % Reduce space between items
\item Collaborated with Dr. Dave Turner to develop ways to enhance the speed and accuracy of the AERI optimal estimation (AERIoe) thermodynamic retrieval algorithm.
\item Performed research evaluating the accuracy of different AERI retrieval methodologies in comparison to other thermodynamic profiling technologies and developed new forecasting methods using the AERI.
\item Mentored one undergraduate student in a research project using AERI retrievals to assess the accuracy of planetary boundary layer parameterization schemes from convective-scale ensemble numerical weather prediction forecasts.$^{1}$
\end{itemize}

{\sl Graduate Research Assistant} \hfill January 2012 -- August 2013 \\
University of Oklahoma, Norman, OK
\begin{itemize} \itemsep 2pt % Reduce space between items
\item Developed and tested statistical thermodynamic retrievals for the AERI, a ground-based passive infrared remote sensing system, by using linear regression, principle component analysis, and neural networks.
\item Provided expertise on AERI thermodynamic retrievals to study nocturnal low-level jet streams during the student run, multi-institutional Lower Atmospheric Boundary Layer Experiment (LABLE) field experiment.
\end{itemize}

{\sl Undergraduate Research Intern} \hfill Fall 2010 \\ 
NOAA Storm Prediction Center, Norman, OK 
\begin{itemize} \itemsep 2pt % Reduce space between items
\item Performed interviews with National Weather Service Warning Coordination Meteorologists to document common severe weather preparedness criteria used by StormReady certified amusement parks and developed additional criteria to be applied to future amusement park StormReady certification.
\item Developed a database of U.S. large venue locations to be integrated into AWIPS for the purpose of heightening forecaster situational awareness.
\item Sat in and participated in forecasting shifts.
\end{itemize} 

{\sl Lead CAREERS Weather Camp Evaluator} \hfill Summer 2010, 2011 \\
Howard University and Jackson State University in Washington, D.C. and Jackson, MS
\begin{itemize} \itemsep 2pt % Reduce space between items
\item Contracted through NCAS (NOAA Center for Atmospheric Sciences) to observe the high school level NCAS Summer Weather Camps and write camp evaluation reports to be used for reporting to the National Science Foundation CAREERS grant (Award \#0914676). Reports included qualitative observations of camp activities, and recommendations for how to evolve and improve future camps.  Additional tasks included providing instructional support to the camp's director, Mike Mogil in order to provide immediate improvements to the camps.
\end{itemize} 

%----------------------------------------------------------------------------------------

\vspace{0.2in} % Some whitespace between sections

%----------------------------------------------------------------------------------------
%	PUBLICATIONS SECTION
%----------------------------------------------------------------------------------------
\section{\centerline{FUTURE PUBLICATIONS}} 

\vspace{15pt} % Gap between title and text
\begin{itemize} \itemsep 2pt % Reduce space between items
%\item Wagner, T.J., K. Cook, and \textbf{W.G. Blumberg}, 2017: High temporal resolution analysis of thermodynamic and kinematic instability associated with the 13 July Nickerson, KS, tornado. \emph{Wea. Forecasting}, in review.
\item \textbf{Blumberg, W.G.}, D. D. Turner, S. Cavallo, J. Basara, and A. Shapiro, 2018: Processes impacting the evolution of moisture and instability in northern Oklahoma during the afternoon to evening transition, \emph{J. Atmos. Sci}, in preparation.
\item \textbf{Blumberg, W.G.}, D. D. Turner, S. Cavallo, J. Basara, A. Shapiro, and J. Gao, 2018: Relationships between the mesoscale evolution of near-surface water vapor and vegetation during the afternoon to evening transition in Oklahoma, \emph{J. Atmos. Sci}, in preparation.
\item Bolton, M. J., \textbf{W.G. Blumberg}, L. K. Ault, H. M. Mogil, S. Haines, 2018: Initial evidence for increased weather salience in autism spectrum conditions, \emph{Weather Clim. Soc.}, in preparation.
\item Bolton, M. J., G. R. Wise, \textbf{W.G. Blumberg}, H. M. Mogil, 2018: Weather messaging concerns for people with color vision differences, \emph{Bull. Amer. Met. Soc.}, in preparation.
\item Bolton, M. J., H. M. Mogil, \textbf{W.G. Blumberg}: An Introduction to the Integration of Meteorology and Psychology. \emph{Book for American Meteorological Society Publishing}, in preparation.
\item Turner, D. D. and \textbf{W.G. Blumberg}, 2018: Improvements to the AERIoe Thermodynamic Profile Retrieval Algorithm, \emph{IEEE JSTARS}, in review.
\end{itemize}

\section{\centerline{PUBLICATIONS AVAILABLE}} 

\vspace{15pt} % Gap between title and text
\begin{itemize} \itemsep 2pt % Reduce space between items
\item \textbf{Blumberg, W.G.}, T. J. Wagner, D. D. Turner, and J. Correia Jr., 2017: \href{http://journals.ametsoc.org/doi/abs/10.1175/JAMC-D-17-0036.1}{Quantifying the accuracy and uncertainty of diurnal thermodynamic profiles and convection indices from the Atmospheric Emitted Radiance Interferometer}, \emph{J. Appl. Met. Climo.}, \textbf{56}, 2747-2766, doi:10.1175/JAMC-D-17-0036.1
\item \textbf{Blumberg, W.G.}, K. T. Halbert, T. A. Supinie, P. T. Marsh, R. L. Thompson, and J. A. Hart, 2017: \href{http://journals.ametsoc.org/doi/abs/10.1175/BAMS-D-15-00309.1}{SHARPpy: An open source sounding analysis toolkit for the atmospheric sciences}, \emph{Bull. Amer. Met. Soc.}, \textbf{98}, 1625-1636, doi:10.1175/BAMS-D-15-00309.1
\item Bolton, M. J., \textbf{W.G. Blumberg}, and H. M. Mogil, 2017: \href{https://osf.io/52nqd}{An Analysis of the Characteristics of Autism Spectrum Conditions for Application to Weather Communication Methods in the Weather Enterprise}. \emph{Open Science Framework.}, doi:10.17605/OSF.IO/52NQD
\item \textbf{Blumberg, W.G.}, D.D. Turner, U. Loehnert, and S. Castleberry, 2015: \href{http://journals.ametsoc.org/doi/abs/10.1175/JAMC-D-15-0005.1}{Ground-based temperature and humidity profiling using spectral infrared and microwave observations.  Part II: Retrieval performance in all-sky conditions}, \emph{J. Appl. Met. Climo.}, \textbf{54}, 2305-2319, doi:10.1175/JAMC-D-15-0005.1
\item Bonin, T. A., \textbf{W.G. Blumberg}, P. M. Klein, and P. B. Chilson, 2015: \href{http://link.springer.com/article/10.1007/s10546-015-0072-2}{Thermodynamic and turbulence characteristics of the Southern Great Plains nocturnal boundary layer under differing turbulent regimes}. \emph{Boundary-Layer Meteorology}, \textbf{157}, 401-420, doi:10.1007/s10546-015-0072-2
%\item Degelia, S., \textbf{W.G. Blumberg}, K. Knopfmeier, D.D Turner, 2014: Evaluation of Storm Scale Ensemble Temperature Forecasts in the Planetary Boundary Layer using the Atmospheric Emitted Radiance Interferometer, \emph{Mon. Wea. Rev.}, in review.
\item Klein P., T.A. Bonin, J.F. Newman,  D.D. Turner, P.B. Chilson, C.E. Wainwright, \textbf{W.G Blumberg}, S. Mishra, M. Carney,  E.P. Jacobsen, and R.K. Newsom, 2014: \href{http://journals.ametsoc.org/doi/abs/10.1175/BAMS-D-13-00267.1}{The 2012 lower atmospheric boundary layer experiment}. \emph{Bull. Amer. Met. Soc.}, \textbf{96}, 1743-1764, doi:10.1175/BAMS-D-132-00267.1
\item \textbf{Blumberg W.G.} and D.D. Turner, 2014: The Dallas-Fort Worth Thermodynamic Profiling Experiment (DFW-TPE). \emph{Report submitted to the National Weather Service}.
\end{itemize}


%----------------------------------------------------------------------------------------

\vspace{0.2in} % Some whitespace between sections

%----------------------------------------------------------------------------------------
%	TEACHING EXPERIENCE SECTION
%----------------------------------------------------------------------------------------

\goodbreak\section{\centerline{TEACHING EXPERIENCE}} 
\vspace{8pt} % Gap between title and text

{\sl Instructor}, University of Oklahoma, Norman, OK \\
METR 3123 - Atmospheric Dynamics 2 \hfill Summer 2018
%\vspace{-5pt} % Reduce space between positions at the same organization
\begin{itemize} \itemsep 2pt % Reduce space between items
\item Led students through discussions and lessons regarding atmospheric dynamics concepts such as balanced flow, circulation, vorticity, Rossby wave behavior, potential vorticity, boundary layer phenomena, and differences between barotropic and baroclinic atmospheres to prepare students for senior level classes.
\item This 8-week course was developed and co-taught with Madeline (Maddie) Frank, an OU graduate student.
\end{itemize}

{\sl Teaching Assistant}, University of Oklahoma, Norman, OK \\
METR 2011 Lab - Introduction to Meteorology 1 Lab \hfill Spring 2018
%\vspace{-5pt} % Reduce space between positions at the same organization
\begin{itemize} \itemsep 2pt % Reduce space between items
\item Developed and provided assignments to improve critical thinking skills, reinforce meteorological concepts, and expose students to various tools (e.g., hand analysis, satellites, numerical weather prediction, programming, sounding analysis) used in the field of meteorology.  
\end{itemize}

{\sl Guest Instructor}, Norman, OK \\[2pt]
CIMMS/Warning Decision Training Division/Weathernews Inc. Training Program  \hfill October 2016
\begin{itemize} \itemsep 2pt % Reduce space between items
\item Trained Japanese visitors from Weathernews Inc. on how to incorporate the SHARPpy program into the severe weather forecasting process.
\item Instructed visiting forecasters in ingredients-based forecasting and hand analysis using real-time data.
\end{itemize}

METR 2603 - Severe and Unusual Weather \hfill Spring 2013
%\vspace{-5pt} % Reduce space between positions at the same organization
\begin{itemize} \itemsep 2pt % Reduce space between items
\item Graded homework and covered classes for the primary instructor.
\end{itemize}

METR 3613 - Measurements Lab \hfill Fall 2011, 2012 
\begin{itemize} \itemsep 2pt % Reduce space between items % Reduce space between items
\item Tasks included leading undergraduate junior meteorology students through lab exercises and grading full lab reports.  
\end{itemize}

METR 1014 - Introduction to Weather and Climate Lab for non-majors \hfill Spring 2011
%\vspace{-5pt} % Reduce space between positions at the same organization
\begin{itemize} \itemsep 2pt % Reduce space between items
\item Developed and presented lessons regarding the weekly lab activity and graded lab reports.
\end{itemize}

{\sl  Informal High School Meteorology Education}  \hfill Summer 2011 -- Summer 2013 \\
Independently Run Long-Distance Education Program
\begin{itemize} \itemsep 2pt % Reduce space between items
\item Developed materials for and provided regular online forecasting and undergraduate-level meteorology education to a variety of high school students across the US.  Students demonstrated their acquired meteorological knowledge by both producing forecasts and performing research$^{2}$ that was presented at AMS conferences.  Mentorship and education were typically conducted using online methods (e.g., Skype and email).
\end{itemize} 

{\sl Director of Development and Shift Leader} \\[2pt]
University of Oklahoma Weather Lab (OWL) \hfill Spring 2009 -- Fall 2012 
\begin{itemize} \itemsep 2pt % Reduce space between items
\item Instructed and led undergraduate students during the weekly forecasting shift to understand meteorological concepts and learn a variety of forecasting techniques.
\item Developed shift leader forecasting training materials.  Taught a workshop session titled ``Perceptions on Quantifiable Words in Forecasting.``
\item Created and distributed a forecasting knowledge survey for tracking student forecasting knowledge development through the OWL program.
\end{itemize} 

{\sl Assistant Instructor}, Jackson, MS and Washington, D.C. \\[2pt]
CAREERS Summer Weather Camps \hfill Summer 2009, 2010, 2011 
\begin{itemize} \itemsep 2pt % Reduce space between items
\item Taught weather related materials to high school students at the Howard University and Jackson State University CAREERS Summer Weather Camp.
\end{itemize} 

{\sl Instructor}, Kicks Karate, Rockville, MD \\[2pt]
Belt Level: 2nd Degree Black Belt \hfill Summer 2004 - Spring 2007 
\begin{itemize} \itemsep 2pt % Reduce space between items
\item Instructed and led karate classes for students of all belt levels from ages 4 to adult.
\end{itemize} 


%\vspace{-6pt} % Reduce space between positions at the same organization
%Department of Psychology \hfill May -- December 1989 
%\begin{itemize} 
%\item Trained engineering students in the development of interpersonal skills. Facilitated group interaction, provided individual feedback via application analysis papers. 
%\end{itemize}
 
%{\sl New York State Department of Mental Health} \hfill September -- December 1989 \\
%Bureau of Management and Program Evaluation, Albany, NY 
%\begin{itemize}
%\item Conducted research to identify key evaluation factors in the statewide investigation of Intensive Care Facilities for disabled persons.
%\end{itemize}

%{\sl Colonie Center Shopping Mall} \hfill January -- June 1989 \\ Albany, NY 
%\begin{itemize} \itemsep -2pt % Reduce space between items
%\item Served as consultant for new management team in techniques for managing change. 
%\item Developed and administered organizational climate survey. 
%\item Facilitated management-employee feedback sessions. 
%\end{itemize}

\goodbreak\section{\centerline{CONFERENCE PROCEEDINGS}} 

\vspace{15pt} % Gap between title and text
\begin{itemize} \itemsep 2pt % Reduce space between items

% Things to add soon...
\item Wagner, T. J., \textbf{W.G. Blumberg}, D. D. Turner, T. A. Bonin, A. Brewer, A. Choukulkar, R. K. Newsom, and V. Wulfmeyer, 2018: Characterizing Water Vapor Advection During LAFE with Thermodynamic and Kinematic Profilers, 23rd Symposium on Boundary Layer and Turbulence, Oklahoma City, OK, USA. 
\item Bonin, T. A., D. D. Turner, R. K. Newsom, L. K. Berg, \textbf{W.G. Blumberg}, A. Behrendt, V. Wulfmeyer, A. Choukulkar, R.M. Banta, Y. Pichugina, C. J. Senff, R. M. Hardesty, and W. A. Brewer, 2018: Observed Characteristics of the Boundary-Layer Evolution during Evening Transitions in North-Central Oklahoma, 23rd Symposium on Boundary Layer and Turbulence, Oklahoma City, OK, USA. 
\item \textbf{Blumberg, W.G.}, and D.D. Turner, 2018: \href{https://ams.confex.com/ams/17MESO/webprogram/Paper320204.html}{Demystifying the 6 o'clock Magic Phenomenon: Environmental Changes during the Southern Great Plains Afternoon to Evening Transition}, San Diego, CA, USA.
\item \textbf{Blumberg, W.G.} and D.D. Turner, 2017: \href{https://ams.confex.com/ams/97Annual/webprogram/Paper313841.html}{Observations of the Afternoon to Evening Transition Occurring Within the Southern Great Plains Severe Convective Environment.} Special Symposium on Severe Local Storms: Observation Needs to advance Research, Prediction, and Communication, Seattle, WA, USA
\item \textbf{Blumberg, W.G.} and T.A. Supinie, 2017: \href{https://ams.confex.com/ams/97Annual/webprogram/Paper315011.html}{Adding Difficulty in Weather Forecasting Challenges to Enhance Learning}. 26th Symposium on Education, Seattle, WA, USA
\item Bolton, M., G. Wise and \textbf{W.G. Blumberg}, 2016: Color Blindness in the Weather Enterprise: Discussion, and a Look at Solutions. National Weather Association's 41st Annual Meeting, Norfolk, VA, USA
\item \textbf{Blumberg, W.G.}, T.J. Wagner, and D.D. Turner, 2016: \href{https://ams.confex.com/ams/96Annual/webprogram/Paper284726.html}{Monitoring the Evolution of Deep Convection Through the Use of Ground-Based Spectral Infrared Thermodynamic Sounders}. 32nd Conference on Environmental Information Processing Technologies, New Orleans, LA, USA.$^{*}$
\item Wagner, T.J., and \textbf{W.G. Blumberg}, 2016: \href{https://ams.confex.com/ams/96Annual/webprogram/Paper289298.html}{Near-continuous Profiling of Atmospheric Stability During Severe Weather Events}.  20th Conference on Integrated Observing and Assimilation Systems for the Atmosphere, Oceans, and Land Surface (IOAS-AOLS), New Orleans, LA, USA.
\item \textbf{Blumberg, W.G.} and D. D. Turner, 2015: Insights Regarding the Use of Ground-Based Spectral Infrared Thermodynamic Sounders in Forecasting Deep Convection. National Weather Association's 40th Annual Meeting, Oklahoma City, OK, USA\textbf{$^{*}$}
\item Bolton, M., and \textbf{W.G. Blumberg}, 2015: \href{http://slideplayer.com/slide/9176419/}{Learning Disorders (LD) in the Meteorological Community: Implications for Communication and Education}. National Weather Association's 41st Annual Meeting, Oklahoma City, OK, USA$^{*}$
\item Halbert, K.T., \textbf{W.G. Blumberg}, and P. Marsh, 2015: \href{https://ams.confex.com/ams/95Annual/webprogram/Paper270233.html}{SHARPpy: Fueling the Python Cult. 5th Symposium on Advances in Modeling and Analysis Using Python}, Phoenix, AZ, USA$^{*}$
\item Yuhas, J.A., \textbf{W.G. Blumberg}, K. Halbert, M. Yalch, T. Ruggiero, E. Mushlitz, M. Stropkay, J. Bailey-Wells, O. Braunstein, and S. Nadler, 2015: \href{https://ams.confex.com/ams/95Annual/webprogram/Paper265787.html}{A University/High School Forecasting Classroom}. 24nd Symposium on Education, Austin, TX, USA$^{2}$
\item Harrison, D., Z. A. Roux, A. McGovern, and \textbf{W.G. Blumberg}, 2015: \href{https://ams.confex.com/ams/95Annual/webprogram/Paper259312.html}{Promoting a Weather Ready Nation Through Serious Games}. 24nd Symposium on Education, Austin, TX, USA
\item Wagner, T. and \textbf{W.G. Blumberg}, 2015: \href{https://ams.confex.com/ams/95Annual/webprogram/Paper260639.html}{Ground-based Infrared Sounders: A New Look at Their Capabilities for Operational Meteorologists}. 19th Conference on Integrated Observing and Assimilation Systems for the Atmosphere, Oceans, and Land Surface (IOAS-AOLS), Phoenix, AZ, USA
\item Turner, D.D. and \textbf{W.G. Blumberg}, 2014: \href{https://www.wmo.int/pages/prog/arep/wwrp/new/wwosc/documents/turner_profiling_vision_wwosc.pdf}{A Future Ground-based Network of Thermodynamic Boundary Layer Profilers: The Infrared Spectrometer Option}. The World Weather Open Science Conference, Montreal, Quebec, Canada
\item \textbf{Blumberg, W.G.} and D.D. Turner, 2014: \href{ftp://ftp.legos.obs-mip.fr/pub/tmp3m/IGARSS2014/abstracts/3989.pdf}{Benefits of Ground-Based AERI Retrievals in Operational Forecasting}. IGARSS, Quebec City, Quebec, Canada
\item Bonin, T.A., P.M. Klein, P.B. Chilson, J.F. Newman, \textbf{W.G. Blumberg}, and D.D. Turner, 2014: \href{https://ams.confex.com/ams/21BLT/webprogram/Paper248197.html}{Analysis of Turbulence and Thermodynamics Associated with Low-Level Jets}. AMS 21st Symposium on Boundary Layers and Turbulence, Leeds, United Kingdom
\item Turner, D.D., \textbf{W.G. Blumberg}, N. Anderson, and A. Dzambo  2014: Characterizing the Structure of the Boundary Layer with AERI and Doppler Lidar. ASR PI Meeting, Potomac, MD, USA
\item \textbf{Blumberg, W.G.}, D. Turner, and P. Klein, 2013: Developing a Statistical Thermodynamic Profiling Retrieval for the AERI. Gordon Research Conference for Radiation and Climate, New London, NH, USA
\item \textbf{Blumberg, W.G.}, P. Klein, and D.D. Turner, 2013: \href{https://ams.confex.com/ams/93Annual/webprogram/Paper217725.html}{Developing a Statistical Thermodynamic Profiling Retrieval for the AERI}. 17th Conference on Integrated Observing and Assimilation Systems for the Atmosphere, Oceans, and Land Surface, Austin, TX, USA 
\item Yuhas, J.A., \textbf{W.G. Blumberg}, K. Halbert, C. Balboni, L. Wallis, and E. Mushlitz, 2013: \href{https://ams.confex.com/ams/93Annual/webprogram/Paper221498.html}{The Concord-Carlisle High School / Oklahoma University Forecasting Group - An experiment in teaching through social media}. 22nd Symposium on Education, Austin, TX, USA$^{2}$
\item Halbert, K.T. and \textbf{W.G. Blumberg}, 2013: \href{https://ams.confex.com/ams/93Annual/webprogram/Paper213476.html}{Forecasting and Analysis in Python: So Easy, a Caveman Can Do It}. 3rd Symposium on Advances in Modeling and Analysis Using Python, Austin, TX, USA$^{2}$
\item Halbert, K.T. and \textbf{W.G. Blumberg}, 2012: \href{https://ams.confex.com/ams/26SLS/webprogram/Paper211764.html}{Utilizing Divergence Tendency in Forecasting Convective Initiation}. 26th Conference on Severe Local Storms, Nashville, TN, USA$^{2}$
\item \textbf{Blumberg, W.G.}, S. Malla, D. V. Morris and H. M. Mogil, 2012: \href{https://ams.confex.com/ams/92Annual/webprogram/Paper202144.html}{Evaluating a Nationwide Summer Weather Camp Program}. 21st Symposium on Education, New Orleans, LA, USA
\item \textbf{Blumberg, W.G.}, K.A. Kloesel, and R. Edwards, 2011: Investigations of hazardous weather preparedness at amusement parks. 2011 National Severe Weather Workshop, Norman, OK, USA
\item \textbf{Blumberg, W.G.}, K.A. Kloesel, and R. Edwards, 2011: \href{https://ams.confex.com/ams/91Annual/webprogram/Paper188077.html}{Investigations of hazardous weather preparedness at amusement parks}. 10th Annual Student Conference, Seattle, WA, USA
\item Bain, A.L., K.D. Sherburn, N.R. Ramsey, and \textbf{W.G. Blumberg}, 2011: \href{https://ams.confex.com/ams/91Annual/webprogram/Paper187924.html}{Oklahoma Weather Lab: An opportunity for operational meteorology for students at the University of Oklahoma}. 10th Annual Student Conference, Seattle, WA, USA
\item Lopes, R., \textbf{W.G. Blumberg}, C. B. Rubin, D. Neal, K. Stevens, 2010: \href{https://www.google.com/url?sa=t&rct=j&q=&esrc=s&source=web&cd=1&ved=0ahUKEwjBvbushpXQAhXslVQKHZlpBMAQFgglMAA&url=https%3A%2F%2Ftraining.fema.gov%2Fhiedu%2F10conf%2Freport%2Fekorikoh%2520-%2520social%2520media.doc&usg=AFQjCNE7FNRfKak5bogh1utz5eF2gBRarg}{Using Social Media in Disaster Preparedness and Response Breakout Session}. 13th Annual Emergency Management Higher Education Conference, Emmitsburg, MD, USA
\end{itemize}
\hfill$^{*}$ indicates paper won an award.
%----------------------------------------------------------------------------------------

\vspace{0.2in} % Some whitespace between sections

%----------------------------------------------------------------------------------------
%	MEMBERSHIPS SECTION
%----------------------------------------------------------------------------------------

\goodbreak\section{\centerline{SOFTWARE DEVELOPED AND MANAGED}} 

\vspace{8pt} % Gap between title and text

{\sl  SHARPpy} - \href{https://github.com/sharppy/SHARPpy}{Sounding and Hodograph Analysis and Research Package in Python}  \hfill April 2014 -- Current \\
Written in: Python
\begin{itemize} \itemsep -2pt % Reduce space between items
\item One of the primary developers of the SHARPpy software package.  This package provides an open source code base of sounding and hodograph analysis routines to the meteorological community and is used internationally.  Development involves collaborations with NWS/SPC employees. Additional co-developers include Kelton Halbert (OU) and Tim Supinie (OU/CAPS).
\end{itemize} 

{\sl MWRoe} - \href{https://github.com/wblumberg/MWRoe}{Microwave Radiometer Optimal Estimation} \hfill Feb 2014 - Current \\
Written in: Python
\begin{itemize} \itemsep -2pt % Reduce space between items
\item Inherited and adapted code from Stephen Castleberry to perform the optimal estimation retrieval method on microwave radiometer data.  Modifications improved optimizing the code to improve the speed and functionality of the code and adapting it to use HATPRO microwave radiometer data.  Code used to retrieve thermodynamic profiles from PECAN microwave radiometers.
\end{itemize} 

{\sl AERIstat} - AERI statistical retrieval \hfill Jan 2012 - August 2013 \\
Written in: Python
\begin{itemize} \itemsep -2pt % Reduce space between items
\item Developed code to create a statistical thermodynamic retrieval for the AERI instrument that could obtain thermodynamic profiles much faster than the current physical retrieval methods.  Developed as part of my master's thesis.
\end{itemize} 

%\vspace{-6pt} % Reduce space between positions at the same organization
%Department of Psychology \hfill May -- December 1989 
%\begin{itemize} 
%\item Trained engineering students in the development of interpersonal skills. Facilitated group interaction, provided individual feedback via application analysis papers. 
%\end{itemize}
 
%{\sl New York State Department of Mental Health} \hfill September -- December 1989 \\
%Bureau of Management and Program Evaluation, Albany, NY 
%\begin{itemize}
%\item Conducted research to identify key evaluation factors in the statewide investigation of Intensive Care Facilities for disabled persons.
%\end{itemize}

%{\sl Colonie Center Shopping Mall} \hfill January -- June 1989 \\ Albany, NY 
%\begin{itemize} \itemsep -2pt % Reduce space between items
%\item Served as consultant for new management team in techniques for managing change. 
%\item Developed and administered organizational climate survey. 
%\item Facilitated management-employee feedback sessions. 
%\end{itemize}

%----------------------------------------------------------------------------------------

\vspace{0.2in} % Some whitespace between sections

\goodbreak\section{\centerline{SELECTED GRAD COURSES}} 

\vspace{15pt} % Gap between title and text
%\vspace{-5pt} % Reduce space between section title and contents
{\sl METR 5803: Advanced Forecasting Techniques}
\hfill Spring 2012 \\ 
Instructor: Dr. Chuck Doswell
\vspace{-4pt} % Some whitespace between sections

{\sl METR 6223: Convective Clouds and Storms}
\hfill Fall 2014 \\ 
Instructor: Dr. Howie Bluestein
\vspace{-4pt} % Some whitespace between sections

{\sl METR 5303: Objective Analysis}
\hfill Fall 2014 \\ 
Instructor: Dr. Fred Carr and Dr. Ming Xue
\vspace{-4pt} % Some whitespace between sections

{\sl METR 5803: Applications of Meteorological Theory to Severe Thunderstorm Forecasting}
\hfill Spring 2015 \\ 
Instructor: Dr. Ariel Cohen, Rich Thompson, and Dr. Steven Cavallo
\vspace{-4pt} % Some whitespace between sections

{\sl METR 5970: Advanced Atmospheric Radiation}
\hfill Spring 2016\\
Instructor: Dr. Dave Turner
\vspace{-4pt} % Some whitespace between sections

%----------------------------------------------------------------------------------------

\vspace{0.2in} % Some whitespace between sections

\goodbreak\section{\centerline{PROFESSIONAL DEVELOPMENT}} 

\vspace{15pt} % Gap between title and text
%\vspace{-5pt} % Reduce space between section title and contents
{\sl AMS Boundary Layer and Turbulence Conference}
\hfill June 2018 \\ 
{\sl AMS Mesoscale Conference}
\hfill July 2017 \\ 
{\sl Forum on Observing the Environment from the Ground Up}
\hfill March 2016 \\ 
{\sl 2nd ITaRS Summer School: "Clouds and Precipitation: Observation and Processes"}
\hfill September 2014 \\ 
{\sl IGARSS Annual Conference}
\hfill July 2014 \\ 
{\sl ARM/ASR Science Meeting}
\hfill March 2014 \\ 
{\sl AMS Monograph on the ARM Program Reviewer}
\hfill October 2013 \\ 
{\sl Gordon Research Conference and Seminar: Radiation \& Climate}
\hfill July 2013 \\ 
{\sl American Meteorological Society Annual Meeting} 
\hfill Yearly 2010 - 2017 \\ 
{\sl American Meteorological Society Severe Local Storms} 
\hfill Nov 2012 \\ 
{\sl AMS Weather, Climate, and Society Journal Peer Reviewer} 
\hfill Nov 2012 \\ 
{\sl WAS*IS Workshop} 
\hfill August 2011 \\ 
{\sl National Severe Weather Workshop 2011} 
\hfill March 2011 \\ 
{\sl Texas Severe Storms Association Conference (TESSA)} 
\hfill March 2009-2011 \\ 
{\sl 13th FEMA Higher Education Conference} 
\hfill June 2010 \\ 

%----------------------------------------------------------------------------------------

\vspace{0.2in} % Some whitespace between sections

%----------------------------------------------------------------------------------------
%	MEMBERSHIPS SECTION
%----------------------------------------------------------------------------------------

\goodbreak\section{\centerline{STUDENTS MENTORED}} 

\vspace{15pt} % Reduce space between section title and contents 

Matt Bolton: long-distance research regarding integrating disabilities into Weather-Ready Nation efforts  \\
Sam Degelia: research assistant as OU meteorology senior, 2014 - 2015$^{1}$ \\
Kelton Halbert: NCAS weather camp alumnus, \href{http://pyaos.johnny-lin.com/?p=1193}{internship from 2011 - 2013}, now \href{http://stormscale.io/author/kthalbert/}{UWisc meteorology grad student}$^{2}$ \\
Celeste Balboni: Concord-Carlisle high school student, 2012 - 2013$^{2}$\\
Lili Wallis: Concord-Carlisle high school student, 2012 - 2013$^{2}$ \\
Emily Mushlitz: Concord-Carlisle high school student, 2012 - 2013$^{2}$ \\
Joe Puma: 2010 NCAS weather camp alumnus, now at \href{http://www.kten.com/story/32829786/joe-puma}{KTEN} \\
Rachel Norris: 2011 NCAS weather camp alumnus, now at \href{http://clasp.engin.umich.edu/people/rbnorris/GSTUDENT}{Univ. Michigan grad student}  \\

%----------------------------------------------------------------------------------------

\vspace{0.2in} % Some whitespace between sections

%----------------------------------------------------------------------------------------
%	MEMBERSHIPS SECTION
%----------------------------------------------------------------------------------------

\goodbreak\section{\centerline{LEADERSHIP \& SERVICE}} 

\vspace{15pt} % Reduce space between section title and contents 

%\begin{center}

The Learningworks Foundation Inc., Board of Directors, Member \\
Serve Moore, OK Tornado Cleanup, Volunteer Weather Forecaster \\
IAVM Severe Weather Preparedness \& Planning For Public Assembly Venues Course, Planning Committee \\
National Weather Camp Program Planning Committee, Member \\
Foot's Forecast, Severe Weather Advisor and Education Consultant \\
Oklahoma Weather Lab, 2010-2011 Director of Development \\
Summer 2012 Oklahoma Mesonet Weather Camp, Invited Guest Speaker \\
University of Oklahoma ``The Big Event``, Event Forecaster \\
National Severe Weather Workshop 2011, Volunteer Weather Briefer \\
Oklahoma Weather Lab, Shift Leader and Forecaster \\
%\end{center}

%----------------------------------------------------------------------------------------

\vspace{0.2in} % Some whitespace between sections

\goodbreak\section{\centerline{AWARDS}} 

\vspace{15pt} % Reduce space between section title and contents 

Second Place Winner, oral presentation category at the EIPT Conference Student Competition, AMS 2016
\indent{Presentation Title:  \textit{Monitoring the Evolution of Deep Convection Through the Use of Ground-Based Spectral Infrared Thermodynamic Sounders. }}

Third Place Winner, poster presentation category at the NWA Conference Grad Student Competition, NWA 2015
\indent{Presentation Title:  \textit{Insights Regarding the Use of Ground-Based Spectral Infrared Thermodynamic Sounders in Forecasting Deep Convection. }}

Alan R. Moller Severe Weather Education and Research College Scholarship, TESSA 2009

%----------------------------------------------------------------------------------------
%	COMPUTER SKILLS SECTION
%----------------------------------------------------------------------------------------

\vspace{0.2in} % Some whitespace between sections

\goodbreak\section{\centerline{SPECIAL SKILLS}}

\vspace{15pt} % Gap between title and text

\begin{itemize} \itemsep -2pt % Reduce space between items
\item Experienced in Python, Linux, R statistical packages, HTML, PHP, LaTeX, and Mathematica \\
\item Completed Institutional Review Board required CITI Course - Social Behavioral Modules, Basic course, and an informational course in ethical human research.
\end{itemize}

%----------------------------------------------------------------------------------------

\vspace{0.2in} % Some whitespace between sections

%----------------------------------------------------------------------------------------


%----------------------------------------------------------------------------------------
%	HONORS SECTION
%----------------------------------------------------------------------------------------

%\section{\centerline{HONORS}} 

%\vspace{-5pt} % Reduce space between section title and contents

%\begin{center}
%Full tuition assistantship, Rensselaer Polytechnic Institute \\
%PSI CHI National Honor Society \\
%PHI THETA KAPPA National Honor Society 
%\end{center}

%----------------------------------------------------------------------------------------

%\vspace{0.2in} % Some whitespace between sections

%----------------------------------------------------------------------------------------
%	INTERESTS SECTION
%----------------------------------------------------------------------------------------

%\section{\centerline{INTERESTS}} 

%\vspace{-5pt} % Reduce space between section title and contents

%\begin{center}
%Jazz piano, swimming, cooking, dancing, gardening
%\end{center} 

%----------------------------------------------------------------------------------------

\end{resume} 
\end{document}